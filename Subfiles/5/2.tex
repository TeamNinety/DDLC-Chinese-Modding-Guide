\documentclass[../../Main.tex]{subfiles}
\begin{document}
\section{特殊效果}

在本节中,我们将学习DDLC二周目的一些特殊效果。

\subsection{撕裂效果}
撕裂效果是DDLC在二周目中最常用的一个错误特效,在默认情况下,撕裂效果会把屏幕撕裂成10份,并产生混乱错乱的效果。它被定义在effects.rpy文件中。

撕裂效果参数的定义如下:
\begin{lstlisting}
screen tear(number=<整数>, ontimeMult=<数字>, offtimeMult=<数字>, offsetMin=<数字>, offsetMax=<数字>, srf=<Surface对象>)
\end{lstlisting}

其中,number参数对应的会将屏幕等分成多个部分,默认为10;ontimeMult 与 offtimeMult 作为撕裂部分来回摆动的时间的乘积因子,默认为1;offsetMin 与 offsetMax 设置了偏移值的最小值与最大值,默认为0与50;srf要求提供一个Surface对象(可以理解成将当前屏幕截图),当参数为None时,会自动生成一个Surface对象,默认为None。

一般的,我们可以直接使用下列代码做到最完美的撕裂错误效果:
\begin{lstlisting}
    show screen tear(20, 0.1, 0.1, 0, 40)
    play sound "sfx/s_kill_glitch1.ogg"
    $ pause(0.25)
    hide screen tear
\end{lstlisting}

\subsection{死机效果}
在上一章节中,我们学习到了bsod.rpy。bsod的参数定义如下:

\begin{lstlisting}
call screen bsod(bsodCode="<字符串>", bsodFile="<字符串>", rsod=<bool值>, chinese_screen=<bool值>)
\end{lstlisting}

其中,bsodCode是一串伪造的错误代码,默认为DDLC\_ESCAPE\_PLAN\_FAILED;bsodFile是导致蓝屏的文件名,默认为libGLESv2.dll。rsod表示是否在将“蓝屏”替换为“红屏”,默认为False;chinese\_screen表示是否在使用中文版本,默认为True。

例如:
\begin{lstlisting}
call screen bsod()
# 或者
call screen bsod("CRITICAL_PROCESS_DIED", "DDLC.exe")
\end{lstlisting}

\subsection{对话修改效果}
在二周目中,我们会经常看到经莫妮卡修改过的对话。要使用这种效果只需要在对话前加入以下代码:
\begin{lstlisting}
    $ style.say_dialogue = style.edited
\end{lstlisting}

\begin{ExtraKnowledge}
    style.say\_dialogue是一个常量,决定了对话的样式。在后面学习了样式的定义后,您甚至可以自己编写一个样式代替原版。
\end{ExtraKnowledge}

若要恢复到正常文本,只需要添加:
\begin{lstlisting}
    $ style.say_dialogue = style.normal
\end{lstlisting}

如下例子:
\begin{lstlisting}
    $ style.say_dialogue = style.edited
    "你爱我吗?"
    $ style.say_dialogue = style.normal
    "...我这是怎么了?"
\end{lstlisting}

\subsection{画面心跳效果}
在二周目中,优里与夏树在第二天争吵时,屏幕会出现心跳效果。要使用这段效果,请在代码中添加:
\begin{lstlisting}
    $ timeleft = 12.453 - get_pos() # 如此莫名其妙的一段代码。
    show noise zorder 3 at noisefade(25 + timeleft) 
    show vignette as flicker zorder 4 at vignetteflicker(timeleft)
    show vignette zorder 4 at vignettefade(timeleft)
    show layer master at layerflicker(timeleft)
\end{lstlisting}

第二行代码会在屏幕上显示噪点。

第三行到第五行代码会让屏幕的四个角落暗淡并晃动。

\subsection{操控历史对话}
在二周目中,有时候优里与玩家的一些异常对话无法在历史里找到或者与玩家在游玩过程中看到的内容不一样。要做到这种效果,我们可以对\_history\_list进行修改达到上述效果。

要删除对话,我们可以使用pop函数删除。如下例:
\begin{lstlisting}
    m "我正在想..."
    m "你希望我来到你的世界吗,[player]?{nw}"
    $ _history_list.pop()
    m "你想要吃小蛋糕吗?"
\end{lstlisting}
在游玩过程中,玩家会正常看到“你希望我来到你的世界吗”这句话,但是在历史对话中,玩家看到的却是:
\begin{lstlisting}
莫妮卡 "我在想..."
莫妮卡 "你想要吃小蛋糕吗?"
\end{lstlisting}

要修改对话,我们可以利用索引修改历史对话。如下例:
\begin{lstlisting}
    s "..."
    s "救救我,[player]!{nw}"
    $ _history_list[-1] = "..."
\end{lstlisting}

在历史对话中,玩家只能看到:
\begin{lstlisting}
纱世里 "..."
纱世里 "..."
\end{lstlisting}
\end{document}