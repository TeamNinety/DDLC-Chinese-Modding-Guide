\documentclass[../../Main.tex]{subfiles}

\begin{document}
\chapter{Python与Ren'Py}
\begin{ChapterGoals}
    \begin{itemize}
        \item 了解Python的基本数据类型;
        \item 学会使用函数与类;
        \item 学会在Ren'Py代码中嵌入Python代码;
        \item 学会使用变量;
        \item 学会进行判断控制;
        \item 学会使用等效语句;
    \end{itemize}
\end{ChapterGoals}

Ren'Py与Python是一个不可分割的整体,在Ren'Py中,许多复杂的操作都需要依靠Python来完成。本章我们将会初步学习如何在Ren'Py中使用Python语句。

\begin{Attention}
    本章只会浅显的介绍Python的用法,若要真正的学习Python,您可以前往 \url{https://www.runoob.com/python3/python3-tutorial.html} 进行更深入的学习。切记,本书不以Python为主。
\end{Attention}

\begin{Warning}
    如果您只是想要简单开发一个Mod,不需要做什么复杂的处理,如只是扩展一下原作剧情,或是给玩家讲述另一个故事,那么您大概率可以跳过本章关于函数、类的学习。但如果您需要开发更复杂的Mod,比如与DDLC有关的AVG游戏,或是像Monika After Story一样的Mod,那么函数与类无疑会方便您后期的开发。

    但是在学习前,您需要注意一些问题。学习Python的难度也比Ren'Py要大许多。而且在Ren'Py中使用Python代码还要格外小心。如在Python中,有几种数据类型需要格外注意。这些数据类型轻则导致代码出现意料之外的运行结果,重则使Ren'Py不稳定崩溃。

    最后,虽然很多Ren'Py语句都有等效的Python代码,但对于Mod来说,不应该使用Python代替Ren'Py代码。Python的作用是方便开发者进行更复杂的操作而不用修改Ren'Py的底层逻辑。但对于大部分的Mod来说,完全可以使用纯Ren'Py代码。更不用说Python代码在一定程度上会增加脚本的复杂度。Python学好了,用好了,就是锦上添花;如果用不好,就会造成开发难度直线上升,Debug及其困难,还会使游戏崩溃给玩家带来负面体验。
\end{Warning}
\subfile{1.tex}
\subfile{2.tex}
\subfile{3.tex}
\subfile{4.tex}
\subfile{5.tex}
\subfile{6.tex}
\end{document}