\documentclass[../../Main.tex]{subfiles}
\begin{document}
\section{使用变量}
\subsection{变量的命名}
变量命名必须遵循以下规则:
\begin{itemize}
    \item 变量名只能包含字母、数字和下划线(\_);
    \item 变量名不能以数字开头;
    \item 变量名不能是Python的关键词(如:def、class、return等);
    \item 变量名区分大小写(如:var与Var是两个不同的变量)。
\end{itemize}

为了提高代码的可读性,避免出现混淆,以下为建议遵守的变量命名规则:
\begin{itemize}
    \item 变量名应具有描述性,能够反映变量的用途或内容。例如,使用player\_score而不是ps;
    \item 对于实例、函数名等,使用下划线命名法,即使用下划线(\_)分隔单词,以提高可读性。例如,使用total\_count而不是 totalcount;
    \item 避免使用过于简短或模糊的变量名,如a、b、data等,除非在非常短的代码段中使用;
    \item 对于类、错误对象,建议使用大驼峰命名法(每个单词的首字母大写且不使用下划线),如MyClass、CustomError而非my\_class、custom\_error。
\end{itemize}

\subsection{python语句块}
关于如何在Python语句块中定义变量,请阅读第\ref{sec:4.3}章
\subsection{define语句}
define语句在初始化时将一个变量赋值。此变量视为一个常量,初始化之后不应再改变。例如:

\begin{lstlisting}
define demo = True
define isActTwo = False
\end{lstlisting}

这段代码的运行效果等于:
\begin{lstlisting}
init python:
    demo = True
    isActTwo = False
\end{lstlisting}

\begin{ExtraKnowledge}
    在define语句中同样可以指定优先级,只需要在define关键词后添加优先级即可,如:define 3 demo = True。
\end{ExtraKnowledge}

define还可以为我们创建一个储存区,只需要将define关键词后的变量改为“储存区.变量”的形式即可。

\subsection{default语句}
default语句会给一个未被定义的变量初始赋值。default语句适合用来定义在游戏过程中会变化的变量。如下例:
\begin{lstlisting}[numbers=none]
default demo = False
\end{lstlisting}

当demo这个变量在游戏开始后没有被定义,则将等价于在start脚本标签中定义demo,且值为False。若在存档加载后没有被定义,则等价于在after\_load魔法标签中定义demo,且值为False。总而言之,若demo这个变量在游戏开始时没有被定义,那么它的值就是False,除非在后来的代码中它的值被改变了。

\subsection{持久化数据}

持久化数据是Ren'Py中的一个储存区,无论用户怎样存档、读档,持久化数据区的数据总是独立于Ren'Py的游戏存档数据的。如DDLC中,对于用户是否在一周目走过了每一条支线、解锁了每一个CG的字典(dict),就存储在持久化数据中,避免因读档存档导致数据消失。简单来说,持久化数据就是不会随着用户存档、读档而改变的数据。

一般来说,在使用DDLC中文Mod模板制作且没有对模版进行设置的游戏,在Ren'Py标准位置里会有一个名为DDLCModTemplateZh的文件夹,里面通常有一个名为persistent的文件,那就是存储持久化数据的存档文件。

持久化数据的用法就和普通变量一样,只不过在前面需要加上“persistent.”前缀。简单来说,持久化数据的语法如下:

\begin{lstlisting}
persistent.<变量名> = <Python 数据类型>
\end{lstlisting}

例如:
\begin{lstlisting}
default persistent.monika_deleted = False
\end{lstlisting}

无论用户如何重启游戏、读档、存档,除非在代码中对持久化数据进行修改或删除persistent文件,persistent.monika\_deleted的值永远都只会是False。
\end{document}